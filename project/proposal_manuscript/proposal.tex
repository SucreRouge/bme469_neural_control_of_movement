\documentclass[a4paper]{article}
\usepackage[a4paper, margin=1.5in]{geometry}
\usepackage[utf8]{inputenc}
\usepackage{amsmath,amsgen,amstext,amsbsy,amsopn,amsfonts}
\usepackage{hyperref}
\usepackage{cite}
\usepackage{fancyhdr}
\pagestyle{fancy}
\rhead{\textit{Titipat Achakulvisut}}
\lhead{BME 469 Project Proposal}

\begin{document}

\section*{Project Proposal}


Understanding how the timing of actions is controlled before
they reach the motor cortex is crucial in movement planning.
Past experiments have suggested that medial prefrontal cortex (dmPFC)
is involved in the timing of actions and the top-down control of motor system
in the motor cortex (MC). This process ocurrs by suppressing responses
during movement delays.

Here we want to replicate the findings by Nandakumar et. al. \cite{narayanan2009delay}
and Bekolay et. al. \cite{bekolay2014spiking} using the nengo simulation system \cite{bekolay2013nengo}.
\cite{narayanan2009delay} decribes neural activity in dmPFC and MC using
time-series Principle Components Analysis (PCA) across neural populations.
They then decribe roles of delay-activity in dmPFC and motor cortex where
they propose the top-down control model between both areas. \cite{bekolay2014spiking}
proposes model to simulate spikes using double-integrator network as a concrete
mechanism that would replicate the results in \cite{narayanan2009delay}.

Concretely, I'll use \textit{nengo} \cite{bekolay2013nengo}, a Python library to simulate spikes trains,
to simulate the model described in \cite{bekolay2014spiking} which
explains the results in \cite{narayanan2009delay}.

\subsection*{Current status}

I have already set up the software tools and simple experiments.
I am currently requesting the experimental data from the authors in
\cite{narayanan2009delay} and they have agreed to provide it.


\bibliography{citations}{}
\bibliographystyle{plain}


\end{document}
